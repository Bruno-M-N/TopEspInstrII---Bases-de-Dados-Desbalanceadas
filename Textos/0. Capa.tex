
% \Title{\texorpdfstring{\large Tópicos Especiais em Instrumentação I}{} \\ Pesquisa de artigos envolvendo fuzzy e as áreas processamento de linguagem natural, ou de agricultura ou de saúde}
%     % \texorpdfstring{\\\large Tópicos Especiais em Instrumentação I}{}}

\Title{ Bases de Dados Desbalanceadas }
    % \texorpdfstring{\\\large Tópicos Especiais em Instrumentação I}{}}

\Author{Bruno Moreira Nabinger $^{1,\dagger}$}
% \Author{Bruno Moreira Nabinger, Lucas Kendrick Dal Castel}

\address{%
$^{1}$ \quad bruno.nabinger@gmail.com}

% \address{E-mail: bruno.nabinger@gmail.com (B.M.N.);  lucaskendrickdalcastel@gmail.com (L.K.D.C.)}

\firstnote{Universidade Federal do Rio Grande do Sul, Departamento de Engenharia Elétrica, Curso de Engenharia Elétrica, Tópicos Especiais em Instrumentação I, Prof. Dr. Alexandre Balbinot}

\datas{20/06/2022}{27/06/2022}


% Deve conter introdução ao trabalho, método, resultados e conclusões – tudo isso de forma resumida em somente 01 parágrafo. De forma geral e simples deve descrever o que foi feito, porque foi feito e resultado(s) do que foi feito. O Resumo deve ser apresentado apenas no formato de 1 parágrafo e contendo no máximo 300 palavras.
\resumo{
O presente trabalho explora técnicas de agrupamento (\textit{clustering}) usando algorítmos k-means e k-medoids (PAM) em conjuntos de dados públicos. No exercício 4 se analisam dados de consumo de energia elétrica, encontrando alguns grupos através do algorítmo k-means. No exercício 11 faz-se a mesma coisa usando-se k-medoids (PAM). No exercicío 15 uma base de dados com dimensões de grãos de trigo de três variedades é usada para avaliar métodos de clusterização usando o k-means. Atinge-se melhores resultados usando-se todos as medidas do trigo com normalização das variáveis nesse caso. Já no exercício 16 a base de dados é um conjunto que dispõe a quantidade de 811 itens vendidos ao longo de 52 semanas, contendo também já a normalização desses volumes de venda, aqui se encontra dificuldade em atribuir melhor valor de k visto que métricas intrínsecas diferentes divergem quanto ao melhor número.
}
% O presente relatório avalia através de dois experimentos a aplicação de conceitos fundamentais na avaliação de medidas obtidas através de algum equipamento de medida tais como: avaliação e propagação incerteza, linearidade/conformidade, erro e sensibilidade. No primeiro experimento foi realizado o ajuste de curvas para bem caracterizar dois termoresistores, enquanto no segundo experimento foram comparadas medidas obtidas através de dois equipamentos distintos para então avaliar e comparar suas incertezas associadas ao medir diferentes grandezas elétricas para determinar a potência elétrica dissipada no circuito.

\abstract{
The present work explores clustering techniques using k-means and k-medoids (PAM) algorithms on public datasets. In exercise 4, data on electricity consumption are analyzed, finding groups using the k-means algorithm. In exercise 11, the same is done using k-medoids (PAM). In exercise 15 a database with geometrical features of wheat grains of three varieties is used to evaluate clustering methods using k-means. Better results are achieved using all wheat measurements with normalization of variables in this case. In exercise 16, the dataset has the amounts of 811 items sold over 52 weeks, also containing the normalization of these sales volumes, it is difficult to assign the better value of k since different intrinsic metrics differ as to the best number. 
}

% \keyword{exemplo; célula de carga;  (no máximo 5 palavras).}
\keyword{cluster; k-means; k-medoids; k-nn; aprendizado não supervisionado.}

