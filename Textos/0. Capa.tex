
% \Title{\texorpdfstring{\large Tópicos Especiais em Instrumentação I}{} \\ Pesquisa de artigos envolvendo fuzzy e as áreas processamento de linguagem natural, ou de agricultura ou de saúde}
%     % \texorpdfstring{\\\large Tópicos Especiais em Instrumentação I}{}}

\Title{ Estudo sobre bases de dados desbalanceadas }
    % \texorpdfstring{\\\large Tópicos Especiais em Instrumentação I}{}}

\Author{Bruno Moreira Nabinger $^{1,\dagger}$}
% \Author{Bruno Moreira Nabinger, Lucas Kendrick Dal Castel}

\address{%
$^{1}$ \quad bruno.nabinger@gmail.com}

% \address{E-mail: bruno.nabinger@gmail.com (B.M.N.);  lucaskendrickdalcastel@gmail.com (L.K.D.C.)}

\firstnote{Universidade Federal do Rio Grande do Sul, Departamento de Engenharia Elétrica, Curso de Engenharia Elétrica, Tópicos Especiais em Instrumentação II, Prof. Dr. Alexandre Balbinot}

\datas{20/06/2022}{27/06/2022}


% Deve conter introdução ao trabalho, método, resultados e conclusões – tudo isso de forma resumida em somente 01 parágrafo. De forma geral e simples deve descrever o que foi feito, porque foi feito e resultado(s) do que foi feito. O Resumo deve ser apresentado apenas no formato de 1 parágrafo e contendo no máximo 300 palavras.
\resumo{O impacto de dados desbalanceados em modelos inteligentes pode ser significativo. $10$ exercícios são abordados com duas bases de dados desbalanceadas diferentes, visando-se empregar algumas das possíveis técnicas para minimizar o problema de dados. Observa-se em algumas situações, como na base de dados desbalanceada do \textit{Exercício viii} a necessidade de realizar o resampling das classes. Nos \textit{Exercício vi} e \textit{Exercício vii}, tatno o upsampling da classe minoritária quanto o downsamplig da classe majoritária apresentaram um desempenho superior nas métricas \textit{recall} e \textit{f-score} considerando-se uma representação adequada de ambas as classes do problema.
}
% O presente relatório avalia através de dois experimentos a aplicação de conceitos fundamentais na avaliação de medidas obtidas através de algum equipamento de medida tais como: avaliação e propagação incerteza, linearidade/conformidade, erro e sensibilidade. No primeiro experimento foi realizado o ajuste de curvas para bem caracterizar dois termoresistores, enquanto no segundo experimento foram comparadas medidas obtidas através de dois equipamentos distintos para então avaliar e comparar suas incertezas associadas ao medir diferentes grandezas elétricas para determinar a potência elétrica dissipada no circuito.

\abstract{
The impact of unbalanced data on intelligent models can be significant. $10$ exercises are discussed with two different unbalanced databases, aiming to employ some of the possible techniques to minimize the data problem. In some situations, as in the unbalanced database in \textit{Exercise viii}, it is necessary to perform class resampling. In \textit{Exercise vi} and \textit{Exercise vii}, both the upsampling of the minority class and the downsampling of the majority class presented a better performance in the metrics \textit{recall} and \textit{f-score} considering an adequate representation of both classes of the problem.
}

% \keyword{exemplo; célula de carga;  (no máximo 5 palavras).}
\keyword{Datasets desbalanceados; Resampling; Penalização; Modelos inteligentes}

