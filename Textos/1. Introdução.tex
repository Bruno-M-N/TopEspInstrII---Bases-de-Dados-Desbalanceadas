% % % INTRODUÇÃO

% % Texto principal.

% % Deve conter uma pequena introdução ao assunto tratado com uma pequena revisão bibliográfica \textbf{(com citações de artigos e livros)} e destacar a descrição dos objetivos (principal(is) e o(s) objetivo(s) secundário(s)) do trabalho tratado no referido relatório. No máximo 2 páginas! Obs ao aluno.: esse modelo de  relatório é baseado na revista científica Sensors da MDPI. As informações textuais foram elaboradas pelo Prof. Dr. Alexandre Balbinot, com algumas adições posteriores do Prof. Dr. Tiago Oliveira Weber.

% % \section{Introdução}

% % Introdução (pequena revisão bibliográfica, objetivos, citações e principalmente a qualidade das citações (referências bibliográficas)

% % contextualizar o nosso trabalho e seu assunto específico

% %estrutura

% %justificar a importância do problema e a relação com a área de instrumentação;

% % deixar claro para o leitor de que se trata de um estudo com base em dados gerados artificialmente baseado em modelo ou simulação.

% %%%%%%%%%%%%%%%%%%%%%%%%%%%%%%%%%%%%%%%%%%%%%%%%%%%%%%%%%%%%%%%%%%%%%%%%%%%%%


% Conforme \cite{zhang_forecasting_2018}, com o desenvolvimento de medidores de eletricidade inteligentes, quantidades expressivas de dados podem ser fornecidas para intervalos diários e horários. Um número significativo de técnicas  tem sido utilizado para prever o consumo elétrico, incluindo redes neurais artificiais (ANN, do inglês \textit{Artificial Neural Networks}), máquinas de vetores de suporte (SVM, do inglês \textit{support-vector machine}), modelos auto-regressivo integrados de médias móveis (ARIMA, do inglês \textit{autoregressive integrated moving average}), modelos de regressão, técnicas de análise de agrupamento de dados (\textit{clustering techniques}) e \textit{empirical mode decomposition} (EMD). 


% % \cite{zhang_forecasting_2018}
% % 1) Data Cleansing
% % Some hourly data were missing and some were in wrong order in the meter readings. To avoid unexpected impacts on the prediction model, missing consumption cells were replaced by the average electrical consumption value of the previous and following hours, and the data were sorted chronologically. In the case of invalid/missing weather conditions or temperature/humidity data in the dataset downloaded from the government Web site, an average tem-perature/humidity value for the previous and following hours was calculated as a substitute for the missing hour. For an invalid weather condition (snow, cloudy, clear, etc.), the cell is filled using the last hour's weather condition.

% Sem a implementação de sensores comportamentais, alguns estudos, conforme \cite{zhang_forecasting_2018}, se concentraram na agregação de padrões de comportamento e abordagens de simulação estocástica para reconhecer e simular padrões comportamentais dos ocupantes com algoritmos de agregação.

% Dessa forma, os clusters tem permitido a obtenção de informações importantes sore os dados, bem como são às vezes utilizados como etapa de pré-processamento.

% Seu papel importante nas abordagem de aprendizado não supervisionado permite a classificação dos dados em categorias. Estas podem, então, ser utilizadas para treinar algoritmos de classificação, como por exemplo o \textit{k-nearest neighbors} (KNN). Na área de consumo de energia elétrica, mencionada anteriormente, o KNN e outros algoritmos classificadoes foram u foi utilizado por 
% \cite{beckel_revealing_cluster_2014} em seu estudo com diversas residências.


% Na Seção Fundamentação Teórica, serão abordados os conceitos necessários para a compreensão do trabalho. A Seção Metodologia Experimental abordará as escolhas impactando na controladora \textit{fuzzy} desenvolvida de forma teórica. Os resultados obtidos serão apresentados e discutidos na Seção Resultados e Discussões. Por fim, na Seção Conclusões, será feita uma conclusão do relatório baseada nos resultados obtidos.


