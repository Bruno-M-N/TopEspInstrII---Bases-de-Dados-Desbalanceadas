\documentclass[journal,article,submit,moreauthors,pdftex]{_Definitions/mdpi} 

\usepackage{mathptmx}

\firstpage{1} 
\makeatletter 
\makeatother
\pubvolume{xx}
\issuenum{1}
\articlenumber{5}
\pubyear{2022}
\copyrightyear{2022}
%\externaleditor{Academic Editor: name}
\history{Received: date; Accepted: date; Published: date}
%\updates{yes} % If there is an update available, un-comment this line

\usepackage{svg}

\usepackage{todonotes}
% \usepackage{enumerate}
% \usepackage{enumitem}
\usepackage{longtable}
\usepackage{multirow}
\usepackage{gensymb}
\usepackage{graphicx}
\usepackage{scalefnt}
\usepackage{graphics}
\usepackage{adjustbox} 
\usepackage{tabularx}  % parada de tabelas (not used)
\usepackage{tabulary}  % parada de tabelas (permite quebrar o texto de uma célula usando 'L')
\usepackage{siunitx}   % Notação Científica
\usepackage{textcomp}
\usepackage{bigstrut}  % parada de tabelas
\usepackage{booktabs}  % parada de tabelas
%  adds PDF support to the landscape environment of package lscape, by setting the PDF /Rotate page attribute. Pages with this attribute will be displayed in landscape orientation by conforming PDF viewers.
\usepackage{pdflscape} 

% \usepackage{lscape}
\usepackage{tikz} % dia
\usepackage{enumitem}
\usepackage{listings}
\usepackage{color}
\usepackage[section]{placeins}
\usepackage{pdfpages}
%\usepackage[opções]{subfigure}
\usepackage{subcaption}
\usepackage{minted}


\usepackage[portuguese]{babel}

\usepackage{csquotes} % Para biblatex
\usepackage[backend=biber,
citestyle=authoryear,
sorting=none,
maxcitenames=2,
% uniquename=false,
uniquename=init,
giveninits=true,
natbib]{biblatex}
% \bibliographystyle{_Definitions/mdpi.sty}
\addbibresource{bibliography.bib}

\newcommand{\MATLAB}{MATLAB\textsuperscript{\textregistered}}
\newcommand{\LabVIEW}{LabVIEW\textsuperscript{TM}}
\newcommand{\Gym}{\textit{Gym}}
\newcommand{\Multisim}{Multisim\textsuperscript{TM}}
\newcommand{\Arduino}{Arduino\textsuperscript{TM}}
\renewcommand{\lstlistingname}{Código}
\usepackage[labelsep=period,labelfont={bf,small}]{caption}
% For tables
%\newcolumntype{L}{>{\centering\arraybackslash$}m{3cm}<{$}} % math-mode version of "l" column type
\newcolumntype{L}{>{$}l<{$}} % math-mode version of "l" column type
\newcolumntype{M}[1]
{>{$}m{#1}<{$}} % math-mode version of "m" column type
\newcolumntype{C}[1]{>{\centering\let\newline\\\arraybackslash\hspace{0pt}}m{#1}}
\newcolumntype{P}[1]{>{\centering\arraybackslash}p{#1}}

\def\stdsize{2.5cm}

\newcolumntype{H}
{m{\stdsize}}

\newcolumntype{B}
{M{\stdsize}}

\lstset{
  language=python,
  basicstyle=\ttfamily\small,
  keywordstyle=\color{blue},
  stringstyle=\color{magenta},
  commentstyle=\color{red},
  extendedchars=true,
  showspaces=false,
  showstringspaces=false,
  showtabs=false,                  
  tabsize=2,
  numbers=right,
  numberstyle=\tiny,
  numbersep=5pt,
  breaklines=true,
  backgroundcolor=\color{green!10},
  breakautoindent=true,
  captionpos=b,
  xleftmargin=0pt,
  keepspaces=true,
}


\usepackage{amsmath}% equation, \numberwithin{equation}, \dfrac{•}{•}

\newcommand{\sen}{\mathop\mathrm{sen}\nolimits}
