%%%%%%%%%%%%%%%%%%%%%%%%%%%%%%%%%%%%%%%%%%%%%%%%%%%%%%%%%%%%%%%%%%%%%%%%%%%%%%%%
% LaTeX

Online LaTeX Equation Editor - Codecogs LaTeX equation
https://latex.codecogs.com/eqneditor/editor.php

Detexify LaTeX handwritten symbol recognition
https://detexify.kirelabs.org/classify.html

Latex - Basic elements for writing a book/thesis -  Mauricio Fernández
https://www.youtube.com/watch?v=Qjp-a2uZWZc

% Tables LaTeX
Create LaTeX tables online – TablesGenerator.com
https://www.tablesgenerator.com/

%%%%%%%%%%%%%%%%%%%%%%%%%%%%%%%%%%%%%%%%%%%%%%%%%%%%%%%%%%%%%%%%%%%%%%%%%%%%%%%%
%

\section{} 

% a subsection with a label. DO NOT put the label if you don't use it afterwards (with a \ref{})
\subsection{}\label{sec:}

% paragraph with title
\paragraph{title of paragraph}
it is good to remember the command 
\footnote{which allows you to make footnotes}

% unordered list
\begin{itemize}
    \item 
\end{itemize}

% ordered list with numbers
\begin{enumerate}
    \item 
\end{enumerate}

% description type list (like a dictionary)
\begin{description}
    \item[titre en gras] description du titre
\end{description}

%Citações
Segundo \citet{vim2012vocabulario}, (citação quando o texto é de apenas um autor) texto texto texto. De acordo com \citet{balbinot2019instrumentacao}, texto texto texto (citação quando o texto é de dois autores). Para mais de dois autores usar et al., por exemplo, \citet{kleijnen_state---art_2005}, texto texto....

% Nesse documento LaTex, há 3 opções para fazer citação. São elas: \cite{} (separa o nome do(s) autor(es) e do ano por vírgula), \citet{} (ano em parentesis) e \citep{} (autor(es) e ano em parentesis).

% Figuras
\begin{figure}[H]
    \centering
    \includegraphics[width=0.5\textwidth]{Figuras/figura_exemplo.png}
    \caption{Fluxograma representando a rotina desenvolvida no (...).\\\textbf{Fonte -} Fonte da figura.}
    \label{fluxograma_N}
\end{figure}

% Tabelas
% NOTA: é possível utilizar o site https://www.tablesgenerator.com/ para gerar tabelas LaTeX  pré-formatadas rapidamente
% Baixe o arquivo "example-Table-to-be-loaded-in-Tables-Generator-com.tgn" na pasta "Tabelas"
% Clique em File e selecione Load table... e depois selecione o arquivo "example-Table-to-be-loaded-in-Tables-Generator-com.tgn"
%
% Adicione um "H" no \begin{table}[] da tabela gerada, de forma a obter \begin{table}[H]
%
%
% No preâmbulo do documento (o que vem antes do \begin{document}), adicione o pacote booktabs com o comando \usepackage{booktabs}, se isso ainda não tiver sido feito, e qualquer outro que apareça em "Result" após clicar no botão "Generate" no site. Nesse documento, booktabs já foi adicionado em "pacotex.tex" na pasta "_Header"
%
% Também é possível copiar tabelas com múltiplas colunas diretamente no site, com um ctrl+c/ ctrl+v
% Em Edit, há um mecanismo de Find and Replace... (Busca e substituição...) que pode ser útil

\begin{table}[]
\centering
\begin{tabular}{@{}ccc@{}}
\toprule
\textbf{Title 1} & \textbf{Title 2} & \textbf{Title 3} \\ \midrule
Entry 1          & Data             & Data             \\
Entry 2          & Data             & Data             \\ \bottomrule
\end{tabular}
\caption{This is a table caption. \\ \textbf{Fonte -} Fonte da tabela.}
\label{tab:my-table}
\end{table}

% % image
% \begin{figure}[H]
%     \centering
%     \includegraphics[width=.8\linewidth]{\imgPath/filename}
%     \caption{Caption}
%     \label{fig:my_label} % to be used only if you use it with the command \ref{}
% \end{figure}

% % table from a .pdf file
% \begin{table}[H]
%     \centering
%     \includegraphics[width=.8\linewidth]{\imgPath/filename}
%     \captionof{Caption}
%     \label{tab:my_label} % to be used only if you use it with the command \ref{}
% \end{table}

