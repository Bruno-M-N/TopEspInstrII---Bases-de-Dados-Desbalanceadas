\begin{table}[H]
    \centering
    \begin{tabular}{c c c} 
        \toprule
        \textbf{Tabuada do 1} & \textbf{Tabuada do 2} & \textbf{Tabuada do 3} \\ [0.5ex] 
        \midrule
        0 & 0 & 0 \\
        \hline
        1 & 2 & 3 \\
        \hline
        2 & 4 & 6 \\
        \hline
        3 & 6 & 9 \\
        \hline
        4 & 8 & 12 \\
        \hline
        5 & 10 & 15 \\
        \bottomrule
    \end{tabular}
    \caption{\label{tabela_tabuadas}Adicionar a legenda da tabela aqui. \\ \textbf{Fonte -} Fonte da tabela.}
\end{table}

% NOTA: é possível utilizar o site https://www.tablesgenerator.com/ para gerar tabelas LaTeX  pré-formatadas rapidamente
% Baixe o arquivo "example-Table-to-be-loaded-in-Tables-Generator-com.tgn" na pasta "Tabelas"
% Clique em File e selecione Load table... e depois selecione o arquivo "example-Table-to-be-loaded-in-Tables-Generator-com.tgn"
%
% Adicione um "H" no \begin{table}[] da tabela gerada, de forma a obter \begin{table}[H]
%
%
% No preâmbulo do documento (o que vem antes do \begin{document}), adicione o pacote booktabs com o comando \usepackage{booktabs}, se isso ainda não tiver sido feito, e qualquer outro que apareça em "Result" após clicar no botão "Generate" no site. Nesse documento, booktabs já foi adicionado em "pacotex.tex" na pasta "_Header"
%
% Também é possível copiar tabelas com múltiplas colunas diretamente no site, com um ctrl+c/ ctrl+v
% Em Edit, há um mecanismo de Find and Replace... (Busca e substituição...) que pode ser útil