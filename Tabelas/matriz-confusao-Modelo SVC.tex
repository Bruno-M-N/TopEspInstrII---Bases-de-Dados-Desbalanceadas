%matriz-confusao-Modelo SVC
% A Tabela \ref{tab: matriz-confusao-Modelo SVC} exibe a Matriz de confusão para o Modelo SVC
% %matriz-confusao-Modelo SVC
% A Tabela \ref{tab: matriz-confusao-Modelo SVC} exibe a Matriz de confusão para o Modelo SVC
% %matriz-confusao-Modelo SVC
% A Tabela \ref{tab: matriz-confusao-Modelo SVC} exibe a Matriz de confusão para o Modelo SVC
% %matriz-confusao-Modelo SVC
% A Tabela \ref{tab: matriz-confusao-Modelo SVC} exibe a Matriz de confusão para o Modelo SVC
% \input{Tabelas/matriz-confusao-Modelo SVC}
% Please add the following required packages to yourdocument preamble:
% \usepackage{booktabs}
% \usepackage{multirow}
\begin{table}[H]
    \centering
    \begin{tabular}{@{}cccc@{}}
    \toprule
    & \textbf{} & \multicolumn{2}{c}{\textbf{Rótulo Verdadeiro}}\\ \midrule
     &  & \begin{tabular}[c]{@{}c@{}}Classe\\ $0$\end{tabular} & \begin{tabular}[c]{@{}c@{}}Classe\\  $1$\end{tabular} \\
    \multirow{2}{*}{\textbf{\begin{tabular}[c]{@{}c@{}}Rótulo\\  Predito\end{tabular}}} & \begin{tabular}[c]{@{}c@{}}Classe\\ $0$\end{tabular} & 72 & 0 \\ 
    & \begin{tabular}[c]{@{}c@{}}Classe\\  $1$\end{tabular} & 5 & 37 \\ \bottomrule
    \end{tabular}
    \caption{Matriz de confusão para o Modelo SVC. \\ \textbf{Fonte -} Autor.}
    \label{tab: matriz-confusao-Modelo SVC}
\end{table}

% Please add the following required packages to yourdocument preamble:
% \usepackage{booktabs}
% \usepackage{multirow}
\begin{table}[H]
    \centering
    \begin{tabular}{@{}cccc@{}}
    \toprule
    & \textbf{} & \multicolumn{2}{c}{\textbf{Rótulo Verdadeiro}}\\ \midrule
     &  & \begin{tabular}[c]{@{}c@{}}Classe\\ $0$\end{tabular} & \begin{tabular}[c]{@{}c@{}}Classe\\  $1$\end{tabular} \\
    \multirow{2}{*}{\textbf{\begin{tabular}[c]{@{}c@{}}Rótulo\\  Predito\end{tabular}}} & \begin{tabular}[c]{@{}c@{}}Classe\\ $0$\end{tabular} & 72 & 0 \\ 
    & \begin{tabular}[c]{@{}c@{}}Classe\\  $1$\end{tabular} & 5 & 37 \\ \bottomrule
    \end{tabular}
    \caption{Matriz de confusão para o Modelo SVC. \\ \textbf{Fonte -} Autor.}
    \label{tab: matriz-confusao-Modelo SVC}
\end{table}

% Please add the following required packages to yourdocument preamble:
% \usepackage{booktabs}
% \usepackage{multirow}
\begin{table}[H]
    \centering
    \begin{tabular}{@{}cccc@{}}
    \toprule
    & \textbf{} & \multicolumn{2}{c}{\textbf{Rótulo Verdadeiro}}\\ \midrule
     &  & \begin{tabular}[c]{@{}c@{}}Classe\\ $0$\end{tabular} & \begin{tabular}[c]{@{}c@{}}Classe\\  $1$\end{tabular} \\
    \multirow{2}{*}{\textbf{\begin{tabular}[c]{@{}c@{}}Rótulo\\  Predito\end{tabular}}} & \begin{tabular}[c]{@{}c@{}}Classe\\ $0$\end{tabular} & 72 & 0 \\ 
    & \begin{tabular}[c]{@{}c@{}}Classe\\  $1$\end{tabular} & 5 & 37 \\ \bottomrule
    \end{tabular}
    \caption{Matriz de confusão para o Modelo SVC. \\ \textbf{Fonte -} Autor.}
    \label{tab: matriz-confusao-Modelo SVC}
\end{table}

% Please add the following required packages to yourdocument preamble:
% \usepackage{booktabs}
% \usepackage{multirow}
\begin{table}[H]
    \centering
    \begin{tabular}{@{}cccc@{}}
    \toprule
    & \textbf{} & \multicolumn{2}{c}{\textbf{Rótulo Verdadeiro}}\\ \midrule
     &  & \begin{tabular}[c]{@{}c@{}}Classe\\ $0$\end{tabular} & \begin{tabular}[c]{@{}c@{}}Classe\\  $1$\end{tabular} \\
    \multirow{2}{*}{\textbf{\begin{tabular}[c]{@{}c@{}}Rótulo\\  Predito\end{tabular}}} & \begin{tabular}[c]{@{}c@{}}Classe\\ $0$\end{tabular} & 72 & 0 \\ 
    & \begin{tabular}[c]{@{}c@{}}Classe\\  $1$\end{tabular} & 5 & 37 \\ \bottomrule
    \end{tabular}
    \caption{Matriz de confusão para o Modelo SVC. \\ \textbf{Fonte -} Autor.}
    \label{tab: matriz-confusao-Modelo SVC}
\end{table}
